\documentclass{article}
%\usepackage[2.5cm]{geometry}
\usepackage{fullpage}
\title{Parallel Image Processing Report}
\date{\today}
\author{Simon Richards (scr52@uclive.ac.nz)}

\begin{document}
\maketitle
\section{Introduction}
An exploration probe belonging to a Chinese mining company has returned from a
two year mission to the Orion Nebula with  approximately 10 million images which
have suffered both motion and gaussian blur. The company has put out a tender
for the restoration of these images using the iterative and computationally
expensive Richardson-Lucy deconvolution algorithm. This report details a brief
investigation of the time and money costs for a potential bid on this contract.

\paragraph{Test Rig}
All performance metrics, unless explicitly stated otherwise, were calculated on
a linux VM running on an AMD Deneb quad-core 3.2GHz cpu with 8 gigabytes of
memory. The effects of running the program on other systems was not explored and
it was assumed that no order-of-magnitude change could be attained. That said,
non-trivial gains could definitely be achieved by upgrading from what is now a
two year old architecture and possibly the overheads imposed by running a
virtual machine.

\section{Implementation}
A single core working with a small point spread function kernel may complete ten
iterations (enough to noticably improve the image) on a single channel,
1024x1024 sized image in 2-10 seconds. Even at one second an image this would
result in the entire dataset being processed in over 115 days, a delay which may
be unnaceptable to the clients. To improve on this time, a double pronged
approach to parallelism has been taken leveraging multiple core CPUs and
multiple computers in a cluster.

\subsection{Class Descriptions}
The task is split into two main classes an io class for reading and writing
images and a filter class capable of efficiently restoring images using the
Richardson-Lucy algorithm.

\subsubsection{ImageQueue}
The first is a producer class which
reads the input directory and pushes the names of all the image files contained in that
directory onto a stack. On request it pops a file name off the stack and reads the
data contained into a buffer using the cfitsio library. FITS is the flexible
image transport system which is popular in scientific fields, especially
astronomy, and is used in this project mostly for the simple way it handles
double precision pixels and dynamic maximum/minimum values.

\subsubsection{DeconvFilter}
The second, and more important, class is the DeconvolutionFilter consumer class.
This class takes a PSF and allocates all the memory space it needs on
construction so that processing a single image is done without any unecessary
overhead. For small kernel sizes (see discussion) the algorithm is as follows:
\begin{verbatim}
FOR iteration = 1 to niter
    temp  = image*psf
    temp  = original_image / temp;
    temp  = temp*psf
    image = temp.*image
ENDFOR
\end{verbatim}
Where / is a per element division, * is the convolution operator and .* is a per
element multiply.

For large kernels the algorithm is equivalent however slightly more involved as
convolution is achieved by performing a fast fourier transform on both matrices,
multiplying the matrices' elements (not the same thing as matrix multiplication)
and then performing an inverse fourier transform. The second option was not
explored past a simple implementation using the FFTW (fastest fourier transform
in the west) library.

\subsubsection{Further Work}
File IO times are minimal in comparison to that that of the deconvolution filter
however we do wish to shave as much time off as possible. Therefore the final
implementation of this project would see a concurrent producer-consumer model
implemented using a semaphore to ensure that a new image is always ready for the
filter process and that the algorithm is never waiting on a disk or network
transfer. For this reason io times are not considered in the calculations seen
in the results section.

\subsection{OpenMP}
OpenMP is an API for Fortran and C/C++ which provides a simple yet powerful
interface to both task and data parallelism. Worker threads are instantiated
cheaply and inline with as little as a single line omp directive such as:
\begin{verbatim}
#pragma omp parallel for private(...) shared(...)
\end{verbatim}

Adding this directive to the convolution function expressed previously resulted
in an execution speed 0.90*(number of cores) times faster than the single
threaded implementation. The other steps of the algorithm were also parallised
in a similar manner however as they were all O(n) the effect of doing this was
negligible.

\subsection{MPI}
Message Passing Interface is a popular standard for communication between
processes in distributed memory systems such as cluster compute set ups.

It enables easy execution and communication between processes

\subsection{Kernel Size}

\section{Results}

\section{Discussion}

\end{document}

